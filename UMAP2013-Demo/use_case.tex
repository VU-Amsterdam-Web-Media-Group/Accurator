\section{Use Case}\label{use_case}

\subsection{Finding candidates for annotation}
\label{finding_candidates}
One of the challenges of nichesourcing is finding candidate annotators that will produce good quality annotations for collection items. This means that besides topical knowledge, properties like availability, willingness to help and being able to share or transfer knowledge are also important. Persons part of a topical community are more likely to have these properties. Being part of a community means an active interest and might mean a willingness to help and share knowledge related to that topic. These topical communities we call niches and manifest themselves, among others, on the Social Web. 

We will perform user studies and analyze social data to understand what identifies a niche community and how to determine whether a person is part of that niche. Next to that we need to understand which properties of a person makes him or her a good candidate to provide qualitative annotations. 

\subsection{Personalizing search for annotation}
\label{personalizing_search}
The challenge for recommender strategies in Accurator is twofold: keep the expertise needed to annotate the artwork in the range of the experts knowledge and yet diversify the suggestions to get high quality annotations for as many distinct artworks as possible. To address these challenges we will investigate the use of content patterns in the Linked Data cloud. Our aim is to develop recommender strategies that use these patterns, resulting in a list of recommendations consisting of diverse artworks.
From a data perspective we aim for diversity in recommendations for the following reasons. Firstly, it makes the handling of the inevitably incomplete data more robust, by providing alternative paths to items. Secondly, using the alternative paths, items can be reached which reside in the long-tail. When experts are able to annotate these long-tail items, they will become more accessible in general.
From a user perspective diversity is also important, we hypothesize that encountering diverse artworks to annotate will help keep the expert motivated. It will also be hard to cover all the possible expertise areas in the user profile from the beginning, so recommending diverse artworks will help to find not yet recorded areas of expertise.

\subsection{Trust mechanisms in annotation}
\label{trust}
In order to obtain quality annotations from the users of the system, we have to tackle issues of determining trust in the users and their contributed information. We address these issues by modelling the user reputation and tracking their expertise across various topics over time. The right model must be chosen to represent the reputation of the user and also their expertise with respect to a particular topic. We intend to use Subjective logic to model the reputation and semantic similarity between topics to track users expertise. The reputation model must allow dynamic updates of the characteristics of the user. Since there is no gold standard for evaluating the contributions from the users, we must develop algorithms for analyzing the quality of the annotations from the users in an open environment. This calls for means to track provenance of the annotation process such as usage of terms from vocabularies by the user, typing speed etc. Research must be carried out to investigate the different metrics which will help in identifying good behaviour of the users. Methods which allow peer review of the annotations must also be studied upon to see how effective they are in the cultural heritage domain. Mechanisms will be investigated as to how to visually represent the different trust levels of the information, which will be addressed in collaboration with research on user interface.

\subsection{User interface design strategies for annotation}
\label{UI}
The professional annotation of artworks is a complex process that requires familiarity with the used classification schemes and (art-)historical expert knowledge. In most cases, both will not be available in candidate users for nichesourcing projects. Therefore, to enable external users to annotate artworks, this process must be broken down into facile tasks that can be solved with little effort and without expert knowledge of classification schemes.
The interface for such a system has to present the task in a straightforward way and motivate the users to contribute their knowledge and time.
To gain a better understanding of how to design such an interface, we investigate what design aspects and underlying mechanisms are responsible for the quality and quantity of tags added by users.
Additionally, we have to find appropriate ways to visualize the trust and personalization aspects in the user interface.
