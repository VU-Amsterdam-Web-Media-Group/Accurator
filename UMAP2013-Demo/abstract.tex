\begin{abstract}
Diversity and profundity of the topics in cultural heritage institutions' collections make experts from outside the institution indispensable for acquiring qualitative annotations. We define the concept of nichesourcing and present challenges in the process of obtaining qualitative annotations from people in these niches. We believe that experts provide better annotations if this process is personalized. We present a framework, called Accurator, that allows to realize and evaluate strategies and applications for personalized nichesourcing.

\keywords{cultural heritage, nichesourcing, annotation framework, qualitative annotations}
\end{abstract}