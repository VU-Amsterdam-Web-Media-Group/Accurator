\begin{abstract}
% A digital collection that can be accessed online, searched and linked to other collections is an important focus for many cultural heritage institutions. 
Diversity and profundity of the topics in cultural heritage institutions collections make experts from outside the institution indispensable to acquiring qualitative annotations. We define the concept of nichesourcing and present the challenges in the process of obtaining qualitative annotations from persons in these niches. Our assumption is that if this process is personalized, we get better annotations from the experts. We present a framework for nichesourcing, called Accurator, that allows to realize and evaluate strategies and applications for personalized nichesourcing.

\keywords{cultural heritage, nichesourcing, annotation framework, qualitative annotations, user interaction}
\end{abstract}